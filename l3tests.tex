\documentclass{article}

\begin{document}

% https://www.alanshawn.com/latex3-tutorial/

\ExplSyntaxOn

\int_new:N \__l_ndu_tmpa_int
\int_new:N \__l_ndu_tmpb_int
\int_new:N \__l_ndu_tmpc_int
\int_new:N \__l_ndu_tmpd_int



\seq_const_from_clist:Nn \g_ndu_units_seq {
	rigsdaler ,
	mark ,
	skilling,
	hvid
}

\tl_new:N \g_ndu_base_unit_tl
\tl_set:Nn \g_ndu_base_unit_tl { hvid }

units:~
\seq_map_inline:Nn { \g_ndu_units_seq } {
	[#1] 
}
\par


\cs_set:Npn \ndu_define_factor:nnn #1#2#3
{
	\int_new:c { g_ndu_factor_#1_#2_int }
	\int_gset:cn { g_ndu_factor_#1_#2_int } { #3 }
%	\int_show:c { g_ndu_factor_#1_#2_int } % DEBUG
}

%\ndu_define_factor:nnn { rigsdaler } { skilling } { 96 }
%\ndu_define_factor:nnn { mark } { skilling } { 16 }
%\ndu_define_factor:nnn { skilling } { skilling } { 1 }

\ndu_define_factor:nnn { rigsdaler } \g_ndu_base_unit_tl { 288 }
\ndu_define_factor:nnn { mark } \g_ndu_base_unit_tl { 48 }
\ndu_define_factor:nnn { skilling } \g_ndu_base_unit_tl { 3 }
\ndu_define_factor:nnn { hvid } \g_ndu_base_unit_tl { 1 }


factor-mark-skilling-int: ( \int_use:c { g_ndu_factor_mark_hvid_int } ) \par



\cs_set:Npn \ndu_factor_div:nnn #1#2#3
{
	\int_div_truncate:nn {#3} { \int_use:c { g_ndu_factor_#1_#2_int } }
}
\cs_set:Npn \ndu_factor_mod:nnn #1#2#3
{
	\int_mod:nn {#3} { \int_use:c { g_ndu_factor_#1_#2_int } }
}


\int_new:N \l_ndu_amount_int
\int_set:Nn \l_ndu_amount_int { 100 }
amount-int: ( \int_use:N \l_ndu_amount_int ) \par


div-test: \int_use:N \l_ndu_amount_int ~hvid~=~ \ndu_factor_div:nnn {rigsdaler} {hvid} {\l_ndu_amount_int} ~rdl \par
div-test: \int_use:N \l_ndu_amount_int ~hvid~=~ \ndu_factor_div:nnn {mark} {hvid} {\l_ndu_amount_int} ~mark \par

%
%div-test~FOO: ( \ndu_factor_div:nnn \l_ndu_amount_int {skilling} { \seq_item:Nn \g_ndu_units_seq {2} } ) \par


\cs_set:Npn \ndu_format_repr:nnnnn #1#2#3#4#5
{
	% parameters
	% #1: unit sequence (seq)
	% #2: base unit (tl)
	% #3: amount (int)
	% #4: unit (tl)
	% #5: seq to store results in
	% scratch vars:
	% a: remaining amount
	% b: current factor
	% c: current amount of that factor
	\seq_clear:N #5

	% init amount * base factor
	\int_set:Nn \__l_ndu_tmpa_int {#3 * \int_use:c { g_ndu_factor_#4_#2_int } }

	% for i, (name, factor) in enumerate(units):
	\seq_map_inline:Nn #1 {
		\int_set:Nn \__l_ndu_tmpb_int { \int_use:c { g_ndu_factor_##1_#2_int } }
		\int_set:Nn \__l_ndu_tmpc_int 0

		% while (amount >= factor):
		\int_while_do:nn {\__l_ndu_tmpa_int >= \__l_ndu_tmpb_int} {
			% amount -= factor
			\int_sub:Nn \__l_ndu_tmpa_int \__l_ndu_tmpb_int
			% result[i] += 1
			\int_incr:N \__l_ndu_tmpc_int
		}
		\seq_put_right:Nx #5 {\int_use:N \__l_ndu_tmpc_int}
		
		% stop when we hit the desired unit depth
		\str_if_eq:NNT {##1} {#4} {
			\seq_map_break:
		}
	}
	
	% TODO format with units
}


%\cs_set:Npn \ndu_amount_to_repr:nnn #1#2#3
%{
%	% parameters
%	% #1: unit sequence
%	% #2: repr
%	% scratch vars:
%}

\seq_new:N \l_ndu_formatted_result_seq

formatted: ( 
	\ndu_format_repr:nnnnn \g_ndu_units_seq \g_ndu_base_unit_tl \l_ndu_amount_int {skilling} \l_ndu_formatted_result_seq
	\seq_use:Nn \l_ndu_formatted_result_seq {.}
 ) \par


\ExplSyntaxOff


\end{document}

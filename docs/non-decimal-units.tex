% !TEX TS-program = LuaLaTeX-shell-escape
\documentclass{article}

\usepackage{makeidx}
\makeindex

\usepackage{needspace}

\usepackage{csquotes}
\usepackage[english]{babel}
\usepackage[
	backend=biber,
	urldate=long,
	date=long,
]{biblatex}
\addbibresource{non-decimal-units.bib}

\usepackage{enumitem}
\usepackage[
	hidelinks,
]{hyperref}

\usepackage{tcolorbox}
\tcbuselibrary{documentation}
\tcbuselibrary{minted}
\tcbuselibrary{breakable}

\definecolor{DarkDefinition}{rgb}{0.392,0.392,0.942}

\tcbset{
	listing engine=minted,
	color definition=DarkDefinition,
	floatplacement=h,
	index colorize=true,
	doc head={
		interior style={fill,color=blue!10},
		boxsep=2pt,
		after skip=0pt,
		nobeforeafter,
	},
	doc raster={
		raster after skip=0pt,
	},
	before doc body={
		\begin{tcolorbox}[
			colback=blue!5,
			colframe=blue!5,
			arc=0pt,
			outer arc=0pt,
			before skip=0pt,
			nobeforeafter,
		]
	},
	after doc body={\end{tcolorbox}},
}

\makeatletter
\AtBeginEnvironment{tcb@manual@entry}{%
	\needspace{10\baselineskip}%
}
\makeatother

\newtcbinputlisting{\unitlisting}[2]{%
	listing file = {../latex/non-decimal-units.#1.tex},
	title = Listing of units loaded with the \docAuxKey{#1} option,
	listing only,
	breakable,
	minted options={fontsize=\small},
	docexample,
}

\setlist[description]{
	leftmargin=5em,
	rightmargin=\leftmargin,
	itemindent=\labelwidth,
	align=right,
	font=\docAuxKey,
	noitemsep,
}

\usepackage{fnpct}

\usepackage{booktabs}

\usepackage{cleveref}

\usepackage[
	danish,
]{non-decimal-units}

\title{Non-Decimal Units for \LaTeX}
\author{Mikkel Eide Eriksen\\\href{mailto:mikkel.eriksen@gmail.com}{\texttt{mikkel.eriksen@gmail.com}}}

\begin{document}

\maketitle

\section{Preface} %%%%%%%%%%%%%%%%%%%%%%%%%%%%%%%%%%%%%%%%%%%%%%%%%%%%%%%%%%%%%%%%%%%%%%%%

Many historical unit systems were non-decimal to simplify mental arithmetic.

For example, 1 \nduName{danish rigsdaler}{0} consists of \nduFactor{danish rigsdaler}{0} \nduName{danish rigsdaler}{1}, which each consist of \nduFactor{danish rigsdaler}{1} \nduName{danish rigsdaler}{2}.

TODO maybe some more historical discussion?

This package enables configuration of such units, to enable display in textual and tabular contexts, as well as perform simple summing.

In order to do this, values are divided into segments, separated by decimal points: The historical Danish monetary value \nduFormatValue{danish rigsdaler}{1.2.3} is entered as \texttt{1.2.3}.

\section{Configuration} %%%%%%%%%%%%%%%%%%%%%%%%%%%%%%%%%%%%%%%%%%%%%%%%%%%%%%%%%%%%%%%%%%

The package is configured in the following manner:

\begin{docCommand}
	{usepackage}
	{\oarg{options}\brackets{non-decimal-units}}

Where \meta{options} may contain one or more of the following unit systems. See \cpageref{units:included} for details.

\begin{description}
\item[british] Currencies
\item[danish] Currencies and areas
\item[german] Currencies
\end{description}

Alternately, one may configure new units via \refCom{nduNewUnit}.

\end{docCommand}

\clearpage
\section{Usage} %%%%%%%%%%%%%%%%%%%%%%%%%%%%%%%%%%%%%%%%%%%%%%%%%%%%%%%%%%%%%%%%%%%%%%%%%%

\subsection{Formatting Values} %%%%%%%%%%%%%%%%%%%%%%%%%%%%%%%%%%%%%%%%%%%%%%%%%%%%%%%%%%%

\begin{docCommand}
	{nduFormatValue}
	{\marg{unit name}\oarg{options}\marg{value}}

Formats \meta{value} according to the setup configured for the \meta{unit name}, as well as any provided \meta{options}. The number of decimal points and the values between them determine how many and which segments are displayed.

Empty segments are skipped.
	
\begin{dispExample*}{
	title=Example usage: \docAuxCommand*{nduFormatValue} macro
}
\nduFormatValue{danish rigsdaler}{1.2.3}\\
\nduFormatValue{danish rigsdaler}{1..}\\
\nduFormatValue{danish rigsdaler}{.2.}\\
\nduFormatValue{danish rigsdaler}{..3}\\
\end{dispExample*}
\end{docCommand}

\clearpage
\subsubsection{Options}
	
\begin{docKeys}[
		doc name = show,
	]{
		{
			doc parameter = {=values},
		},
		{
			doc parameter = {=values and units},
			doc description = initially \texttt{values and units},
		},
		{
			doc parameter = {=units},
		},
	}

Changes which information is included in the expansion.

Only those segments with a value will be included, which means that \docAuxKey*{show}=\docValue*{units} can be used to list the segment units.

\begin{dispExample}
\nduFormatValue{danish hartkorn}
	[show=units]
	{0.0.0.0.0}

\nduFormatValue{danish hartkorn}
	[show=units]
	{0.0...}
\end{dispExample}
\end{docKeys}

\begin{docKey}
	{segment separator}
	{=\meta{...}}
	{initially configured by the unit}

Changes the separator between each segment.

\begin{dispExample}
\nduFormatValue{danish hartkorn}[
		show=values,
		segment separator=.
	]
	{1.2.3.4}

\nduFormatValue{danish rigsdaler}
	[segment separator={---}]
	{1.2.3}
\end{dispExample}
\end{docKey}

\clearpage
\subsection{Tabular Data} %%%%%%%%%%%%%%%%%%%%%%%%%%%%%%%%%%%%%%%%%%%%%%%%%%%%%%%%%%%%%%%%

In order to align values in a tabular context, the \docAuxCommand*{nduAlignedHeader} and \docAuxCommand*{nduAlignedValue} macros wrap each segment in a \docAuxCommand*{makebox} of equal width. 

All segments will be included in the headers and cells, whether they contain a value or not.

\begin{docCommand}
	{nduAlignedHeader}
	{\marg{unit name}\oarg{options}}
	Formats the unit symbols in boxes suitable for a header. See \cpageref{units:new} for configuration of symbols.
\end{docCommand}

\begin{docCommand}
	{nduAlignedValue}
	{\marg{unit name}\oarg{options}\marg{value}}
	See \refCom{nduFormatValue} for possible arguments.
\end{docCommand}

\begin{dispExample*}{
	title=Example usage: \docAuxCommand*{nduAlignedHeader} and \docAuxCommand*{nduAlignedValue} macros
}
\begin{tabular}{r r}
	\toprule
	  & \nduAlignedHeader{danish rigsdaler} \\
	\midrule
	a & \nduAlignedValue{danish rigsdaler}{1.2.3} \\
	b & \nduAlignedValue{danish rigsdaler}{100..} \\
	c & \nduAlignedValue{danish rigsdaler}{.1.} \\
	\bottomrule
\end{tabular}
\end{dispExample*}

\clearpage
\subsubsection{Options}

\begin{docKey}
	{aligned value width}
	{=\meta{length}}
	{initially \texttt{5em}}

Changes the width of each segment.

\begin{dispExample*}{
	title=Example usage: \docAuxKey*{aligned value width} key
}
\begingroup
\nduset{
	aligned value width=3em,
}
\begin{tabular}{r r}
	\toprule
	& \nduAlignedHeader{danish rigsdaler} \\
	\midrule
	a & \nduAlignedValue{danish rigsdaler}{1.2.3} \\
	b & \nduAlignedValue{danish rigsdaler}{100..} \\
	c & \nduAlignedValue{danish rigsdaler}{.1.} \\
	\bottomrule
\end{tabular}
\endgroup
\end{dispExample*}
\end{docKey}

\clearpage
\subsection{Summing Values} %%%%%%%%%%%%%%%%%%%%%%%%%%%%%%%%%%%%%%%%%%%%%%%%%%%%%%%%%%%%%%

Values can be accumulated in a named sum in two ways, either manually via the \docAuxCommand*{nduAddToSum} macro, or automatically via the \docAuxKey*{sum to} key.

\begin{docCommands}[]{
	{
		doc name = nduAddToSum,
		doc parameter = \marg{unit name}\oarg{options}\marg{sum name}\marg{value}
	},
	{
		doc name = nduFormatSum,
		doc parameter = \marg{unit name}\oarg{options}\marg{sum name}
	}
}

The arguments of \docAuxCommand*{nduAddToSum} are identical to those of the \refCom{nduFormatValue} macro, except for the addition of the \meta{sum name} argument, under which the sum will be accumulated. It does not expand to any output.

The \docAuxCommand*{nduFormatSum} macro takes the \meta{sum name} and formats it according to the current settings.

Both may be further configured via the \meta{options}.

\begin{dispExample*}{
	title=Example usage: \docAuxCommand*{nduAddToSum} and \docAuxCommand*{nduFormatSum} macros
}
\nduAddToSum{danish rigsdaler}{example 1}{0.0.10}
\nduAddToSum{danish rigsdaler}{example 1}{0.0.8}
\nduAddToSum{danish rigsdaler}{example 1}{0.2.0}
\nduAddToSum{danish rigsdaler}{example 1}{0.5.1}
\nduFormatSum{danish rigsdaler}{example 1} % = 1.2.3
\end{dispExample*}

The same sum can also be displayed as aligned values:

\begin{dispExample}
\nduAlignedHeader{danish rigsdaler}\\
\nduAlignedSum{danish rigsdaler}{example 1} % = 1.2.3
\end{dispExample}
\end{docCommands}

\clearpage
\subsubsection{Options}

\begin{docKey}
	{sum to}
	{=\meta{name}}
	{initially empty}

Setting this key will cause all uses of \docAuxCommand*{nduFormatValue} and \docAuxCommand*{nduAlignedValue} in the current group to be summed under the given name.

\begin{dispExample*}{
	title=Example usage: \docAuxKey*{sum to} key
}
\begingroup
\nduset{
	aligned value width=3em,
	sum to=example 2
}
\begin{tabular}{r r}
	\toprule
	& \nduAlignedHeader{danish rigsdaler} \\
	\midrule
	a & \nduAlignedValue{danish rigsdaler}{1.2.3} \\
	b & \nduAlignedValue{danish rigsdaler}{100..} \\
	c & \nduAlignedValue{danish rigsdaler}{.1.} \\
	\bottomrule
	total & \nduAlignedSum{danish rigsdaler}{example 2} \\ % = 101.3.3
\end{tabular}
\endgroup
\end{dispExample*}

Sums are global and remain accessible outside the group:
\begin{dispExample}
\nduFormatSum{danish rigsdaler}{example 2}
\end{dispExample}

Adding an additional 15 \nduName{danish rigsdaler}{2} to the existing sum gives:
\begin{dispExample}
\nduAddToSum{danish rigsdaler}{example 2}{0.0.15}
\nduFormatSum{danish rigsdaler}{example 2} % = 101.4.2
\end{dispExample}
\end{docKey}

\subsection{Accessing Information About Units} %%%%%%%%%%%%%%%%%%%%%%%%%%%%%%%%%%%%%%%%%%%

\begin{docCommand}
	{nduName}
	{\marg{unit name}\marg{segment}}
	Expands to the name of the the given segment of the unit.
\end{docCommand}

\begin{docCommand}
	{nduFactor}
	{\marg{unit name}\marg{segment}}
	Expands to the factor of the the given segment of the unit, ie. how many of the underlying segment the given segment consists of.

\begin{dispExample}
That is, 1 \nduName{danish rigsdaler}{0} consists of
\nduFactor{danish rigsdaler}{0} \nduName{danish rigsdaler}{1}.
\end{dispExample}
\end{docCommand}

\subsection{Creating New Units} %%%%%%%%%%%%%%%%%%%%%%%%%%%%%%%%%%%%%%%%%%%%%%%%%%%%%%%%%%

\label{units:new}
If the included units are not suitable, more can be created. Pull requests are also welcome at \url{https://github.com/mikkelee/latex-units}.

\begin{docCommand}
	{nduNewUnit}
	{\marg{unit name}\marg{key/value pairs}}
	
Units can have up to 5 segments, numbered \meta{0-4}. The left-most segment, that is, the \emph{top} or \emph{root} segment, is numbered \texttt{0}.

The numeral part of the below key paths \docAuxKey*{segment 0/} can be any integer up to \texttt{4}, ie. \docAuxKey*{segment 4/}. The internal number of segments is determined by how many symbol keys are created.

See below for available settings.

\end{docCommand}

\begin{docCommand}
	{nduNewMacro}
	{\marg{unit name}\marg{key/value pairs}\marg{control sequence}}

	It is possible to create shortcut macros for commonly used \meta{unit name}s with optional default settings.

	These macros take the same arguments as the full \refCom{nduFormatValue} macro, except without the first argument (ie. the name of the unit).

\end{docCommand}

\clearpage
\subsubsection{Options}

\begin{docKey}
	{segment separator}
	{=\meta{...}}
	{initially \texttt{\~{}}}
	
	When displaying a value, this string will be inserted between each segment.
\end{docKey}

\begin{docKey}
	[segment 0]{name}
	{=\meta{name}}
	{no default, initially undefined}

	Useful for giving the full name of the segment's unit, but unused except by \refCom{nduName}.
\end{docKey}

\begin{docKey}
	[segment 0]{symbol}
	{=\meta{symbol}}
	{no default, initially undefined}

	Configures a symbol displaying the unit. This is used in \docAuxCommand{nduAlignedHeader} and is also available via \docAuxCommand{nduSymbol} when defining the \refKey*{/segment 0/display} (see below).
	
	The symbols are also used internally to calculate how many segments are possible.
\end{docKey}

\begin{docKey}
	[segment 0]
	{display}
	{=\marg{prefix}\marg{suffix}}
	{initially \texttt{\{\}\{ \docAuxCommand*{nduSymbol}\}}}

	When displaying a value, the segments will be wrapped between the \meta{prefix} and \meta{suffix}.
	
	The macro \docAuxCommand{nduSymbol} is available here to show the symbol configured for the segment.
	
	The default is to use the symbol as prefix, but can be overriden if necessary.
\end{docKey}

\begin{docKey}
	[segment 0]
	{factor}
	{=\meta{integer}}
	{no default, initially undefined}
	
	The factor of a segment is how many of the underlying segment the given segment consists of.
	
	This is used when summing values, in order to calculate the correct segment values.
	
	Can be accessed via \refCom{nduFactor}.
\end{docKey}

These keys can of course also be set temporarily in \refCom{nduFormatValue}

\begin{dispExample}
\nduFormatValue{danish rigsdaler}
	[segment 1/symbol=Mk.]
	{.9.}

\nduFormatValue{danish rigsdaler}
	[segment 0/display={}{ Rigsdaler og}]
	{1.2.3}

\nduFormatValue{danish rigsdaler}[
		segment separator={---},
		segment 0/display={(}{)},
		segment 1/display={[}{]},
		segment 2/display={\{}{\}},
	]
	{1.2.3}
\end{dispExample}

\begin{docKey}
	{create macro named}
	{=\meta{control sequence}}
	{no default, initially empty}
	
	Units may provide a default shortcut macro, for example the \docValue*{danish rigsdaler} unit configures \docAuxCommand*{rdl}.

\begin{dispExample*}{
	sidebyside,
	righthand width=1.75cm,
}
\rdl{2.3.}
\end{dispExample*}
\end{docKey}

\clearpage
\subsection{Included Units} %%%%%%%%%%%%%%%%%%%%%%%%%%%%%%%%%%%%%%%%%%%%%%%%%%%%%%%%%%%%%%

\label{units:included}
On the following pages are the units included with the package.

\unitlisting{british}

\clearpage
\unitlisting{danish}

\unitlisting{german}

\printindex  %%%%%%%%%%%%%%%%%%%%%%%%%%%%%%%%%%%%%%%%%%%%%%%%%%%%%%%%%%%%%%%%%%%%%%%%%%%%%

\end{document}